\documentclass[]{book}
\usepackage{lmodern}
\usepackage{amssymb,amsmath}
\usepackage{ifxetex,ifluatex}
\usepackage{fixltx2e} % provides \textsubscript
\ifnum 0\ifxetex 1\fi\ifluatex 1\fi=0 % if pdftex
  \usepackage[T1]{fontenc}
  \usepackage[utf8]{inputenc}
\else % if luatex or xelatex
  \ifxetex
    \usepackage{mathspec}
  \else
    \usepackage{fontspec}
  \fi
  \defaultfontfeatures{Ligatures=TeX,Scale=MatchLowercase}
  \newcommand{\euro}{€}
\fi
% use upquote if available, for straight quotes in verbatim environments
\IfFileExists{upquote.sty}{\usepackage{upquote}}{}
% use microtype if available
\IfFileExists{microtype.sty}{%
\usepackage{microtype}
\UseMicrotypeSet[protrusion]{basicmath} % disable protrusion for tt fonts
}{}
\usepackage{hyperref}
\PassOptionsToPackage{usenames,dvipsnames}{color} % color is loaded by hyperref
\hypersetup{unicode=true,
            pdftitle={R Programming for Research},
            pdfauthor={Brooke Anderson and Rachel Severson},
            pdfsubject={Colorado State University, ERHS 535},
            pdfborder={0 0 0},
            breaklinks=true}
\urlstyle{same}  % don't use monospace font for urls
\usepackage{natbib}
\bibliographystyle{apalike}
\usepackage{color}
\usepackage{fancyvrb}
\newcommand{\VerbBar}{|}
\newcommand{\VERB}{\Verb[commandchars=\\\{\}]}
\DefineVerbatimEnvironment{Highlighting}{Verbatim}{commandchars=\\\{\}}
% Add ',fontsize=\small' for more characters per line
\usepackage{framed}
\definecolor{shadecolor}{RGB}{248,248,248}
\newenvironment{Shaded}{\begin{snugshade}}{\end{snugshade}}
\newcommand{\KeywordTok}[1]{\textcolor[rgb]{0.13,0.29,0.53}{\textbf{{#1}}}}
\newcommand{\DataTypeTok}[1]{\textcolor[rgb]{0.13,0.29,0.53}{{#1}}}
\newcommand{\DecValTok}[1]{\textcolor[rgb]{0.00,0.00,0.81}{{#1}}}
\newcommand{\BaseNTok}[1]{\textcolor[rgb]{0.00,0.00,0.81}{{#1}}}
\newcommand{\FloatTok}[1]{\textcolor[rgb]{0.00,0.00,0.81}{{#1}}}
\newcommand{\ConstantTok}[1]{\textcolor[rgb]{0.00,0.00,0.00}{{#1}}}
\newcommand{\CharTok}[1]{\textcolor[rgb]{0.31,0.60,0.02}{{#1}}}
\newcommand{\SpecialCharTok}[1]{\textcolor[rgb]{0.00,0.00,0.00}{{#1}}}
\newcommand{\StringTok}[1]{\textcolor[rgb]{0.31,0.60,0.02}{{#1}}}
\newcommand{\VerbatimStringTok}[1]{\textcolor[rgb]{0.31,0.60,0.02}{{#1}}}
\newcommand{\SpecialStringTok}[1]{\textcolor[rgb]{0.31,0.60,0.02}{{#1}}}
\newcommand{\ImportTok}[1]{{#1}}
\newcommand{\CommentTok}[1]{\textcolor[rgb]{0.56,0.35,0.01}{\textit{{#1}}}}
\newcommand{\DocumentationTok}[1]{\textcolor[rgb]{0.56,0.35,0.01}{\textbf{\textit{{#1}}}}}
\newcommand{\AnnotationTok}[1]{\textcolor[rgb]{0.56,0.35,0.01}{\textbf{\textit{{#1}}}}}
\newcommand{\CommentVarTok}[1]{\textcolor[rgb]{0.56,0.35,0.01}{\textbf{\textit{{#1}}}}}
\newcommand{\OtherTok}[1]{\textcolor[rgb]{0.56,0.35,0.01}{{#1}}}
\newcommand{\FunctionTok}[1]{\textcolor[rgb]{0.00,0.00,0.00}{{#1}}}
\newcommand{\VariableTok}[1]{\textcolor[rgb]{0.00,0.00,0.00}{{#1}}}
\newcommand{\ControlFlowTok}[1]{\textcolor[rgb]{0.13,0.29,0.53}{\textbf{{#1}}}}
\newcommand{\OperatorTok}[1]{\textcolor[rgb]{0.81,0.36,0.00}{\textbf{{#1}}}}
\newcommand{\BuiltInTok}[1]{{#1}}
\newcommand{\ExtensionTok}[1]{{#1}}
\newcommand{\PreprocessorTok}[1]{\textcolor[rgb]{0.56,0.35,0.01}{\textit{{#1}}}}
\newcommand{\AttributeTok}[1]{\textcolor[rgb]{0.77,0.63,0.00}{{#1}}}
\newcommand{\RegionMarkerTok}[1]{{#1}}
\newcommand{\InformationTok}[1]{\textcolor[rgb]{0.56,0.35,0.01}{\textbf{\textit{{#1}}}}}
\newcommand{\WarningTok}[1]{\textcolor[rgb]{0.56,0.35,0.01}{\textbf{\textit{{#1}}}}}
\newcommand{\AlertTok}[1]{\textcolor[rgb]{0.94,0.16,0.16}{{#1}}}
\newcommand{\ErrorTok}[1]{\textcolor[rgb]{0.64,0.00,0.00}{\textbf{{#1}}}}
\newcommand{\NormalTok}[1]{{#1}}
\usepackage{longtable,booktabs}
\setlength{\parindent}{0pt}
\setlength{\parskip}{6pt plus 2pt minus 1pt}
\setlength{\emergencystretch}{3em}  % prevent overfull lines
\providecommand{\tightlist}{%
  \setlength{\itemsep}{0pt}\setlength{\parskip}{0pt}}
\setcounter{secnumdepth}{5}

\title{R Programming for Research\\\vspace{0.5em}{\large Colorado State University, ERHS 535}}
\author{Brooke Anderson and Rachel Severson}
\date{2016-11-12}
\usepackage{booktabs}

\usepackage{longtable}
\usepackage[bf,singlelinecheck=off]{caption}

\usepackage{framed,color}
\definecolor{shadecolor}{RGB}{248,248,248}

\renewcommand{\textfraction}{0.05}
\renewcommand{\topfraction}{0.8}
\renewcommand{\bottomfraction}{0.8}
\renewcommand{\floatpagefraction}{0.75}

\ifxetex
  \usepackage{letltxmacro}
  \setlength{\XeTeXLinkMargin}{1pt}
  \LetLtxMacro\SavedIncludeGraphics\includegraphics
  \def\includegraphics#1#{% #1 catches optional stuff (star/opt. arg.)
    \IncludeGraphicsAux{#1}%
  }%
  \newcommand*{\IncludeGraphicsAux}[2]{%
    \XeTeXLinkBox{%
      \SavedIncludeGraphics#1{#2}%
    }%
  }%
\fi

\makeatletter
\newenvironment{kframe}{%
\medskip{}
\setlength{\fboxsep}{.8em}
 \def\at@end@of@kframe{}%
 \ifinner\ifhmode%
  \def\at@end@of@kframe{\end{minipage}}%
  \begin{minipage}{\columnwidth}%
 \fi\fi%
 \def\FrameCommand##1{\hskip\@totalleftmargin \hskip-\fboxsep
 \colorbox{shadecolor}{##1}\hskip-\fboxsep
     % There is no \\@totalrightmargin, so:
     \hskip-\linewidth \hskip-\@totalleftmargin \hskip\columnwidth}%
 \MakeFramed {\advance\hsize-\width
   \@totalleftmargin\z@ \linewidth\hsize
   \@setminipage}}%
 {\par\unskip\endMakeFramed%
 \at@end@of@kframe}
\makeatother

\renewenvironment{Shaded}{\begin{kframe}}{\end{kframe}}

\newenvironment{rmdblock}[1]
  {
  \begin{itemize}
  \renewcommand{\labelitemi}{
    \raisebox{-.7\height}[0pt][0pt]{
      {\setkeys{Gin}{width=3em,keepaspectratio}\includegraphics{images/#1}}
    }
  }
  \setlength{\fboxsep}{1em}
  \begin{kframe}
  \item
  }
  {
  \end{kframe}
  \end{itemize}
  }
\newenvironment{rmdnote}
  {\begin{rmdblock}{note}}
  {\end{rmdblock}}
\newenvironment{rmdcaution}
  {\begin{rmdblock}{caution}}
  {\end{rmdblock}}
\newenvironment{rmdimportant}
  {\begin{rmdblock}{important}}
  {\end{rmdblock}}
\newenvironment{rmdtip}
  {\begin{rmdblock}{tip}}
  {\end{rmdblock}}
\newenvironment{rmdwarning}
  {\begin{rmdblock}{warning}}
  {\end{rmdblock}}

\usepackage{makeidx}
\makeindex

\urlstyle{tt}

% Redefines (sub)paragraphs to behave more like sections
\ifx\paragraph\undefined\else
\let\oldparagraph\paragraph
\renewcommand{\paragraph}[1]{\oldparagraph{#1}\mbox{}}
\fi
\ifx\subparagraph\undefined\else
\let\oldsubparagraph\subparagraph
\renewcommand{\subparagraph}[1]{\oldsubparagraph{#1}\mbox{}}
\fi

\begin{document}
\maketitle

\cleardoublepage\newpage\thispagestyle{empty}\null
\cleardoublepage\newpage\thispagestyle{empty}
\begin{center}
To ...
\end{center}

\frontmatter

{
\setcounter{tocdepth}{1}
\tableofcontents
}
\chapter*{Online course book, ERHS
535}\label{online-course-book-erhs-535}
\addcontentsline{toc}{chapter}{Online course book, ERHS 535}

This is the online book for Colorado State University's ERHS 535 \emph{R
Programming for Research} course. This book includes course information,
course notes, links to download pdfs of lecture slides, in-course
exercises, homework assignments, and vocabulary lists for quizzes for
this course. Because this is my first semester teaching the course with
this online book, it will be evolving throughout the semester, as we get
to new material.

\mainmatter

\chapter*{Course information}\label{course-information}
\addcontentsline{toc}{chapter}{Course information}

\href{https://github.com/geanders/RProgrammingForResearch/raw/master/slides/CourseOverview.pdf}{Download}
a pdf of the lecture slides covering this topic.

\section{Course overview}\label{course-overview}

This document provides the course notes for Colorado State University's
ERHS 535 course for Fall 2016. The course offers in-depth instruction on
data collection, data management, programming, and visualization, using
data examples relevant to academic research.

\section{Time and place}\label{time-and-place}

This course meets in Room 120 of the Environmental Health Building on
Mondays and Wednesdays, 10:00 am--12:00 pm. Exceptions to these meeting
times are:

\begin{itemize}
\tightlist
\item
  There will be no meeting on Wednesday, Aug. 31.
\item
  There will be no meeting on Labor Day (Monday, Sept. 5).
\item
  To make up for missing class on Aug. 31, we will have a supplemental
  class on Friday, Sept. 9, 10:00 am--12:00 pm. You \textbf{will not}
  lose attendance points if you cannot attend this class, but
  \textbf{will} be responsible for the material covered.
\item
  There are no course meetings the week of Thanksgiving.
\end{itemize}

\section{Detailed schedule}\label{detailed-schedule}

Here is a more detailed view of the schedule for this course for Fall
2016:

\begin{tabular}{l|l|l|l}
\hline
Dates & Level & Lecture content & Graded items\\
\hline
Aug. 22, 24 & Preliminary & R Preliminaries & \\
\hline
Aug. 29 & Basic & Entering and cleaning data & \\
\hline
Sept. 7, Sept. 9* & Basic & Exploring data & Quiz (W)\\
\hline
Sept. 12, 14 & Basic & Reporting data results & Quiz (M), HW \#1 (W)\\
\hline
Sept. 19, 21 & Basic & Reproducible Research & Quiz (M)\\
\hline
Sept. 26, 28 & Intermediate & Entering and cleaning data & Quiz (M), HW \#2 (W)\\
\hline
Oct. 3, 5 & Intermediate & Exploring data & Quiz (M)\\
\hline
Oct. 10, 12 & Intermediate & Reporting data results & Quiz (M), HW \#3 (W)\\
\hline
Oct. 17, 19 & Intermediate & Reproducible Research & Quiz (M), Group choices (M)\\
\hline
Oct. 24, 26 & Advanced & Entering and cleaning data & Quiz (M), Project proposal (M), HW \#4 (W)\\
\hline
Oct. 31, Nov. 2 & Advanced & Exploring data & \\
\hline
Nov. 7, 9 & Advanced & Reporting data results & HW \#5 (W)\\
\hline
Nov. 14, 16 & Advanced & Mapping in R & \\
\hline
Nov. 28, 30 & Advanced & Package development 1 & HW \#6 (W)\\
\hline
Dec. 5, 7 & Advanced & Package development 2 & Project draft (M)\\
\hline
Week of Dec. 12 &  & Group presentations & Final project (M)\\
\hline
\end{tabular}

\section{Grading}\label{grading}

Course grades will be determined by the following five components:

\begin{tabular}{l|r}
\hline
Assessment component & Percent of grade\\
\hline
Final group project & 30\\
\hline
Weekly in-class quizzes, weeks 3-10 & 25\\
\hline
Homework & 25\\
\hline
Attendance and class participation & 10\\
\hline
Weekly in-course group exercises & 10\\
\hline
\end{tabular}

\subsection{Attendance and class
participation}\label{attendance-and-class-participation}

Because so much of the learning for this class is through interactive
work in class, it is critical that you come to class. Out of a possible
10 points for class attendance, you will get:

\begin{itemize}
\tightlist
\item
  \textbf{10 points} if you attend all classes
\item
  \textbf{8 points} if you miss one class
\item
  \textbf{6 points} if you miss two classes
\item
  \textbf{4 points} if you miss three classes
\item
  \textbf{2 points} if you miss four classes
\item
  \textbf{0 points} if you miss five or more classes
\end{itemize}

You can get two extra credit attendance points (i.e., make up for a
missed class) by attending either the seminar that Yihui Xie will give
on Sept. 23 at 4:00 pm for the Statistics Department in Weber 237 to the
short course he will give at 10:00-11:00 am in Weber 223H. (You are
welcome to attend both, but can only get a maximum of two extra credit
attendance points.)

\subsection{Weekly in-course group
exercises}\label{weekly-in-course-group-exercises}

Part of each class will be spent doing in-course group exercises. Ten
points of your final grade will be based on your participation in these
exercises. As long as you are in class and participate in these
exercises, you will get full credit for this component. If you miss a
class, to get credit towards this component of your grade, you will need
to turn in a one-page document describing what you learned from doing
the in-course exercise on your own time. All in-class exercises are
included in the online course book at the end of the chapter on the
associated material.

\subsection{In-class quizzes}\label{in-class-quizzes}

You will have eight total in-class quizzes. You will have one for each
of the Week 2--10 class meetings. There will be \emph{at least} 10
questions per quiz. You will get 1/3 point for each correct answer. If
you do the math, you can get full credit for this if you get at least
75\% of your answers right. You can not get more than the maximum of 25
points for this component-- once you reach 25 points on quizzes, you
will have achieved full credit for the quiz component of the course
grade.

All quiz questions will be multiple choice, matching, or some other form
of ``close-answered'' question (i.e., no open-response-style questions).
You can not make up a quiz for a class period you missed. You can still
get full credit on your total possible quiz points if you miss a class,
but it means you will have to work harder and get more questions right
for days you are in class.

Because grading format for these quizzes allows for you to miss some
questions and still get the full quiz credit for the course, I will not
ever re-consider the score you got on a previous quiz, give points back
for a wrong answer on a poorly-worded question, etc. However, if a lot
of people got a particular question wrong, I will be sure to cover it in
the next class period. Also, especially if a question was poorly worded
and caused confusion, I will work a similar question into a future
quiz-- in addition to the 10 guaranteed questions for that quiz-- so
every student will have the chance to get an extra 1/3 point of credit
for the question.

The ``Vocabulary'' appendix of our online book has the list of material
for which you will be responsible for this quiz. Most of the functions
and concepts will have been covered in class, but some may not. You are
responsible for going through the list and, if there are things you
don't know or remember from class, learning them. To do this, you can
use help functions in R, Google, StackOverflow, books on R, ask a
friend, and any other resource you can find.

In general, using R frequently in your research or other coursework will
help you to prepare and do well on these quizzes.

\subsection{Homework}\label{homework}

There will be six homework assignments, starting a few weeks into the
course and then due approximately every two weeks (see the detailed
schedule in the online course book for exact due dates).

Homeworks should be done individually. You will get many chances to work
with others during in-course exercises and your final group project, but
these homeworks should be a chance to assess how well you understand and
can use the course material on your own.

Homeworks will be graded for correctness, but some partial credit will
be given for questions you try but fail to answer correctly. If you
can't completely do a required task, be sure to show and explain what
you tried to do to complete it.

Homework is due by the start of class on the due date. Your grade will
be reduced by 10 points for each day it is late, and will receive no
credit if it is late by over a week.

\subsection{Final group project}\label{final-group-project}

You will do the final group project in groups of 2--3. The final product
will be a statistical blog post-style article of 1,500 words or less and
an accompanying Shiny web application. Come up with an interesting
question you'd love to get the answer to that you think you can find
data to help you answer. You will need to use the data you find, and R,
to write your article. The final product will be a Word document created
from an RMarkdown file and an accompanying Shiny web application.

Here are some articles to give you an idea of the style and content for
this project:

\begin{itemize}
\tightlist
\item
  \href{http://www.statslife.org.uk/culture/1892-does-christmas-really-come-earlier-every-year}{Does
  Christmas come earlier each year?}
\item
  \href{http://hilaryparker.com/2013/01/30/hilary-the-most-poisoned-baby-name-in-us-history/}{Hilary:
  the most poisoned baby name in US history}
\item
  \href{http://fivethirtyeight.com/datalab/every-guest-jon-stewart-ever-had-on-the-daily-show/}{Every
  Guest Jon Stewart Ever Had On ``The Daily Show''}
\item
  \href{http://fivethirtyeight.com/features/should-travelers-avoid-flying-airlines-that-have-had-crashes-in-the-past/}{Should
  Travelers Avoid Flying Airlines That Have Had Crashes in the Past?}
\item
  \href{http://fivethirtyeight.com/features/billion-dollar-billy-beane/}{Billion-Dollar
  Billy Beane}
\end{itemize}

You will have in-class group work time during weeks 10--15 to work on
this. This project will also require some work with your group outside
of class. You will be able to get feedback from me through weekly
informal written reports in these weeks. I will also provide feedback
and help during the in-class group work time.

The final group project will be graded with A through F, with the
following point values (out of 30 possible):

\begin{itemize}
\tightlist
\item
  \textbf{30 points} for an A
\item
  \textbf{25 points} for a B
\item
  \textbf{20 points} for a C
\item
  \textbf{15 points} for a D
\item
  \textbf{10 points} for an F
\end{itemize}

If you turn nothing in, you will get \textbf{0 points}.

\subsubsection{Final presentation}\label{final-presentation}

\begin{itemize}
\tightlist
\item
  In total, the group's presentation should last 25-30 minutes. There
  will then be 5 minutes for questions.
\item
  Split the presentation up into three parts: (1) the main presentation
  (10-15 minutes), (2) a tutorial-style discussion of how you used R to
  do the project (10 minutes) and (3) an overview of your Shiny app (5
  minutes).
\item
  The main presentation part should include the following sections:

  \begin{itemize}
  \tightlist
  \item
    \textbf{Research question}: In one sentence, what is the main thing
    you were trying to figure out?
  \item
    \textbf{Introduction}: Why did you decide to do this project? You
    must convince us that your project answers an interesting and
    important question.
  \item
    \textbf{Methods}: Where did you find data to answer the question?
    Why are they appropriate? Are there any limitations we should keep
    in mind? How did you investigate the data to try to answer your
    question? This should not include R code (save that for the tutorial
    part), but rather should use language like ``To determine if
    \ldots{} was associated with \ldots{}, we measured the correlation
    \ldots{}''. It's fine for this project if the Methods are fairly
    simple (``We investigated the distribution of \ldots{} using
    boxplots \ldots{}'', ``We took the mean and interquartile range of
    \ldots{}'', ``We mapped state-level averages of \ldots{}'', etc.).
    Why do you choose to use the Methods you used? Why do you think
    they're appropriate and useful for your project?
  \item
    \textbf{Results}: What did you find out? Most of these slides should
    be figures or tables. Discuss your interpretation of your results as
    you present them. Ideally, you should be able to show your main
    results in about 4 slides, with one figure or table per slide.
  \item
    \textbf{Conclusions}: So what? How do your results compare with what
    other people have found out about your research question? Based on
    what you found, are there now other things you want to check out?
  \end{itemize}
\item
  The tutorial part should include the following sections:

  \begin{itemize}
  \tightlist
  \item
    \textbf{Overview of your approach in R}: Step us through a condensed
    version of how you did your project
  \item
    \textbf{Interesting packages / techniques}: Spend a bit more time on
    any parts that you found particularly interesting or exciting. Were
    there packages you used that were helpful that we haven't talked
    about in class? Did you find out how to do anything that you think
    other students could use in the future? Did you end up writing a lot
    of functions to use? Did you have an interesting way of sharing code
    and data among your group members?
  \item
    \textbf{Lessons learned}: If you were to do this project again from
    scratch, what would you do differently? Were there any big wrong
    turns along the way? Did you find out how to do something late in
    the project that would have saved you time if you'd started using it
    earlier?
  \end{itemize}
\end{itemize}

\subsubsection{Final report}\label{final-report}

The final report should not exceed 1,500 words. You should aim for no
more than five figures and tables.

In addition to the good examples linked above, you can find another
example of the type of document we're looking for
\href{http://fivethirtyeight.com/features/the-20-most-extreme-cases-of-the-book-was-better-than-the-movie/}{here}
from FiveThirtyEight. This example would have received an excellent
score if it had been turned in for this class because it is clearly and
engagingly written, it presents figures and tables that directly help to
answer its main question and that are clearly explained and attractively
presented, and its author has convinced me that this is an interesting
question worth reading an article about. Notice that it is not very
long, only has three figures and tables, and uses fairly simple
analysis.

I will assess the final report on the following criteria:

\begin{itemize}
\tightlist
\item
  Is it written with correct spelling and grammar?
\item
  Is it very clear what your over-arching research question is?
\item
  Have you made a convincing case that this is an important or
  interesting problem? You could meet this criterion even by convincing
  me that this is a problem that just one of you is passionate about (as
  an example, see
  \href{http://hilaryparker.com/2013/01/30/hilary-the-most-poisoned-baby-name-in-us-history/}{here}).
\item
  Are the data that you chose to use reasonable for answering the
  question? Have you explained any caveats or limitations to the data
  that I should keep in mind when interpreting your results? As an
  example of how to do this for an analysis with secondary (imperfect)
  data, see how
  \href{http://fivethirtyeight.com/features/the-20-most-extreme-cases-of-the-book-was-better-than-the-movie/}{this
  post} handles describing the data it uses, particularly in footnotes 1
  and 3 and the sentences in the main text that correspond to them.
\item
  Have you explained the way you analyzed the data clearly enough that I
  think that I could reproduce your analysis if I had your data? Have
  you explained a bit why your method of analyzing the data is
  appropriate for your question? Have you let me know about major
  caveats or limitations related to the methods of analysis you're
  using?
\item
  Have you presented figures and / or tables with results that help
  answer your main research question? Is it clear what each is showing
  and how I should interpret it? (For a nice example of explaining how
  to interpret results, see footnote 4
  \href{http://fivethirtyeight.com/features/the-20-most-extreme-cases-of-the-book-was-better-than-the-movie/}{here}.)
  Have you explained and interpreted your main results in the text? Have
  you pointed out any particularly interesting observations (interesting
  outliers, for example)?
\item
  When I'm finished with your article, do I have more insight into your
  research question than when I started?
\item
  If you include a quote or a figure from an outside source, you
  \textbf{must} include a full reference for it. Otherwise, I am okay
  with you doing referencing more in a blog-post style. That is, if you
  are repeating another person's ideas or findings, you must reference
  it, but you may use a web link rather than writing out full
  references. You do not need to include references of any type for
  standard analysis techniques (for example, you would not need to
  include a reference from a Stats book if you are fitting a regression
  model).
\end{itemize}

\subsubsection{Shiny app}\label{shiny-app}

Finally, you should create a Shiny app to provide an interactive
visualization related to your research question. Expectations for that
are:

\begin{itemize}
\tightlist
\item
  It should work.
\item
  It should include text that is clearly written, without grammatical
  errors or typos. Any graphics are easily to interpret and follow some
  of the principles of good graphics covered in class.
\item
  It includes at least two rendered outputs (e.g., one plot and one
  table; two plots).
\item
  It is self-contained-- in other words, a user shouldn't need to read
  your report or hear your explain the app to understand it. It should
  include enough information on the app for a user to figure out how to
  use the app and interpret the output.
\item
  It is publicly available online. The easiest way to do this is to
  create a free account at shinyapps.io and publish to that. Only one
  person in the group needs to publish the app.
\end{itemize}

\section{Course set-up}\label{course-set-up}

Please be sure you have the latest version of R and RStudio installed.
Also, be sure to sign up for a free GitHub account.

\section{Helpful books for learning
R}\label{helpful-books-for-learning-r}

There are three publishers that are leaders in good books for learning
R:

\begin{itemize}
\tightlist
\item
  \href{http://shop.oreilly.com/category/browse-subjects/programming/r.do?sortby=publicationDate\&page=all}{O'Reilly}
\item
  \href{https://www.nostarch.com}{No Starch Press}
\item
  \href{http://www.springer.com/generic/search/results?SGWID=4-40109-24-653415-0\&sortOrder=relevance\&searchType=ADVANCED_CDA\&language=en\&searchScope=editions\&resultStart=1\&queryText=\%22+R+\%22}{Springer}
\end{itemize}

Some particular books you might want to check out:

\begin{itemize}
\tightlist
\item
  \href{http://r4ds.had.co.nz}{R for Data Science}
\item
  \href{http://discovery.library.colostate.edu/Record/.b40129810}{R for
  Dummies}
\item
  \href{http://discovery.library.colostate.edu/Record/.b40438880}{R in a
  Nutshell}
\item
  \href{http://discovery.library.colostate.edu/Record/.b36840282}{R
  Cookbook}
\item
  \href{http://www.amazon.com/R-Graphics-Cookbook-Winston-Chang/dp/1449316956/ref=sr_1_1?ie=UTF8\&qid=1440997472\&sr=8-1\&keywords=r+graphics+cookbook}{R
  Graphics Cookbook}
\item
  \href{http://discovery.library.colostate.edu/Record/.b44709365}{A
  Beginner's Guide to R}
\item
  \href{https://leanpub.com/u/rdpeng}{Roger Peng's Leanpub books}
\end{itemize}

Books that other students have found useful include:

\begin{itemize}
\tightlist
\item
  Introductory R by Robert J. Knell
\end{itemize}

\appendix


\chapter{Appendix A: Vocabulary}\label{appendix-a-vocabulary}

You will be responsible for knowing the following functions and
vocabulary for the weekly quizzes.

\section{Week 1 (Quiz 1)}\label{week-1-quiz-1}

\begin{itemize}
\tightlist
\item
  \texttt{c()}
\item
  \texttt{data.frame()}
\item
  \texttt{dim()}
\item
  \texttt{ncol()}
\item
  \texttt{nrow()}
\item
  \texttt{head()}, option \texttt{n\ =}
\item
  \texttt{read.csv}, options \texttt{head\ =}, \texttt{skip\ =},
  \texttt{nrow\ =}
\item
  \texttt{{[}...{]}}, \texttt{{[}...,\ ...{]}}
\item
  \texttt{getwd()}
\item
  \texttt{setwd()}, including \texttt{setwd("\textasciitilde{}")}
\item
  \texttt{list.files()}
\item
  \texttt{install.packages()}
\item
  \texttt{library()}
\item
  \texttt{\textless{}-}
\item
  \texttt{=}
\item
  \texttt{subset()}
\item
  \texttt{length()}
\item
  open source software
\item
  ``free as in beer''
\item
  ``free as in speech''
\item
  CRAN
\item
  GitHub
\item
  R packages
\item
  R working directory
\item
  How to download a csv file from GitHub
\item
  Nate Silver
\item
  FiveThirtyEight
\item
  Grading policies for the course
\item
  Course requirements / policies for in-class quizzes and weekly journal
  entries
\item
  Style rules for naming R objects
\item
  Difference between R and RStudio
\item
  Vectors
\item
  Dataframes
\item
  Note: Pay attention in the course notes and exercise to where the code
  uses quotation marks and where it does not-- this will help you in the
  quiz
\end{itemize}

\section{Week 2 (Quiz 2)}\label{week-2-quiz-2}

\begin{itemize}
\tightlist
\item
  \texttt{source()}
\item
  \texttt{setwd()}, including \texttt{setwd("\textasciitilde{}")},
  \texttt{setwd("..")}, \texttt{setwd("..\textbackslash{}..")}
\item
  \texttt{list.files()}, option \texttt{path\ =}
\item
  functions in the \texttt{read.table()} family, including
  \texttt{read.csv()} and \texttt{read.delim()}. What are defaults of
  the \texttt{sep\ =} and \texttt{dec\ =} options for each? For all, the
  options \texttt{header\ =}, \texttt{sep\ =}, \texttt{as.is\ =},
  \texttt{na.strings\ =}, \texttt{nrows\ =}, \texttt{skip\ =}, and
  \texttt{col.names\ =}.
\item
  The tidyverse
\item
  functions in the \texttt{read\_*} family (e.g., \texttt{read\_csv})
\item
  Advantages of the \texttt{read\_*} family of functions compared to
  their base R analogues (the \texttt{read.table} functions)
\item
  \texttt{paste()}, option \texttt{sep\ =}
\item
  \texttt{paste0()}
\item
  \texttt{readxl} package and its \texttt{read\_excel()} function
\item
  \texttt{haven} package and its \texttt{read\_sas()} function
\item
  \texttt{\$}
\item
  \texttt{class()}
\item
  \texttt{str()}
\item
  \texttt{as.Date()}, option \texttt{format\ =}
\item
  \texttt{lubridate} functions, include \texttt{ymd}, \texttt{ymd\_hm},
  and \texttt{mdy}
\item
  \texttt{range()}
\item
  \texttt{dplyr} package
\item
  \texttt{rename()}
\item
  \texttt{mutate()}
\item
  \texttt{arrange()}
\item
  \texttt{\%\textgreater{}\%}, advantages of piping
\item
  \texttt{filter()}
\item
  Reading in data from either a local or online flat file
\item
  \texttt{save()}, option \texttt{file\ =}
\item
  \texttt{load()}
\item
  \texttt{rm()}
\item
  \texttt{ls()}
\item
  Main types of vector classes in R: character, numeric, factor, date,
  logical
\item
  Which classes of vectors don't always look like numbers, but R assigns
  an underlying numeric value to? (Hint: This include the logical class,
  which R saves with an underlying number, with TRUE = 1 and FALSE = 0.)
\item
  Common abbreviations for telling R date formats (e.g., ``\%m'',
  ``\%y'')
\item
  Common logical expressions to use in \texttt{filter()}
\item
  relative pathnames
\item
  absolute pathnames
\item
  delimited files
\item
  fixed width files
\item
  R script file (How would you make a new one? What file extension would
  it have? Why is it important to use? How do you run code from a script
  file in RStudio?)
\item
  What kinds of data can be read into R?
\item
  How to read flat files of data that are online directly into R if they
  are on:

  \begin{itemize}
  \tightlist
  \item
    A ``http:'' site
  \item
    A ``https:'' site
  \end{itemize}
\item
  When you might want to save an R object as a \texttt{.RData} file and
  when (and why) you might not want to
\end{itemize}

\section{Week 3 (Quiz 3)}\label{week-3-quiz-3}

\begin{itemize}
\tightlist
\item
  \texttt{data()} (with and without the name of a dataset as an option)
\item
  \texttt{library()} (with and without an argument in the parentheses)
\item
  common addition for all these plotting functions: \texttt{ggtitle},
  \texttt{xlab}, \texttt{ylab}, \texttt{xlim}, \texttt{ylim}
\item
  \texttt{aes} function and common aesthetics, including \texttt{color},
  \texttt{shape}, \texttt{x}, \texttt{y}, and \texttt{fill}
\item
  Some common geoms: \texttt{geom\_histogram}, \texttt{geom\_points},
  \texttt{geom\_lines}, \texttt{geom\_boxplot()} (both for a single
  numeric variable and for a numeric vector stratified by a factor)
\item
  \texttt{ggpairs()} from the \texttt{GGally} package
\item
  \texttt{range()}
\item
  \texttt{min()}
\item
  \texttt{max()}
\item
  \texttt{mean()}
\item
  \texttt{median()}
\item
  \texttt{table()}
\item
  \texttt{cor()}, both for two variables in a dataframe, and to get the
  correlation matrix for several variables in a dataframe
\item
  \texttt{summary()}, as applied to: different classes of vectors
  (numeric, factor, logical), dataframes, \texttt{lm} objects, and
  \texttt{glm} objects
\item
  \texttt{lm()}, \texttt{data=} option
\item
  \texttt{glm()}, options \texttt{data=}, \texttt{family=}
\item
  Functions to apply to a \texttt{lm} or \texttt{glm} object:
  \texttt{summary()}, \texttt{coef()}, \texttt{residuals()},
  \texttt{fitted()}, \texttt{plot()}, \texttt{abline()}
\item
  The following elements that you can pull from the \texttt{summary} of
  a \texttt{lm} call: \texttt{summary(mod\_1)\$call},
  \texttt{summary(mod\_1)\$coef}, \texttt{summary(mod\_1)\$r.squared},
  \texttt{summary(mod\_1)\$cov.unscaled}
\item
  How to create a logical vector and how to use one to (1) index a data
  frame and (2) count the number of times a certain condition is true in
  a vector
\item
  What the bang operator (\texttt{!}) does to a logical operator
\item
  What to do if you want to apply a summary statistic function to a
  vector with missing values (you do not need to know every option name
  for all the functions, just know that you would need to include an
  option like \texttt{na.rm=} or \texttt{use=}, and that you can use the
  help file for a function to figure out the option call for that
  function).
\item
  The following about object-oriented programming: In R, it means that
  some functions, like \texttt{summary()}, will do different things
  depending on what type of object you call it on.
\item
  The basic structure of regression formulae in R (for example,
  \texttt{y\ \textasciitilde{}\ x1\ +\ x2})
\item
  Difference between using \texttt{lm()} and \texttt{glm()} to fit a
  linear regression model
\item
  Difference between the code you would use to fit a linear, Poisson, or
  logistic model using \texttt{glm()}
\end{itemize}

\section{Week 4 (Quiz 4)}\label{week-4-quiz-4}

\begin{itemize}
\tightlist
\item
  Guidelines for good graphics
\item
  Data density / data-to-ink ratio
\item
  Small multiples
\item
  Edward Tufte
\item
  Hadley Wickham
\item
  Where to put the \texttt{+} in ggplot statements to avoid problems
  (ends of lines instead of starts of new lines)
\item
  Can you save a ggplot object as an R object that you can reference
  later? If so, how would you add elements on to that object? How would
  you print it when you were ready to print the graph to your RStudio
  graphics window?
\item
  \texttt{geom\_hline()}, \texttt{geom\_vline()}
\item
  \texttt{geom\_text()}
\item
  \texttt{facet\_grid()}, \texttt{facet\_wrap()}
\item
  \texttt{grid.arrange()} from the \texttt{gridExtra} package
\item
  \texttt{ggthemes} package, including \texttt{theme\_few()} and
  \texttt{theme\_tufte()}
\item
  Setting point color for \texttt{geom\_point()} both as a constant (all
  points red) and as a way to show the level of a factor for each
  observation
\item
  \texttt{size}, \texttt{alpha}, \texttt{color}
\item
  Re-naming and re-ordering factors
\item
  \textbf{Note:} If you read this and find and bring in an example of a
  ``small multiples'' graph (from a newspaper, a website, an academic
  paper), you can get one extra point on this quiz
\end{itemize}

\section{Week 5 (Quiz 5)}\label{week-5-quiz-5}

\begin{itemize}
\tightlist
\item
  Reproducible research, including what it is and advantages to aiming
  to make your research reproducible
\item
  R style guidelines on variable names, \texttt{\textless{}-} vs.
  \texttt{=}, line length, spacing, semicolons, commenting, indentation,
  and code grouping
\item
  Markup languages (concept and examples)
\item
  Basic conventions for Markdown (bold, italics, links, headers, lists)
\item
  Literate programming
\item
  What working directory R uses for code in an .Rmd document
\item
  Basic syntax for RMarkdown chunks, including how to name them
\item
  Options for RMarkdown chunks: \texttt{echo}, \texttt{eval},
  \texttt{messages}, \texttt{warnings}, \texttt{include},
  \texttt{fig.width}, \texttt{fig.height}, \texttt{results}
\item
  Difference between global options and chunk options, and which takes
  precendence
\item
  What inline code is and how to write it in RMarkdown
\item
  How to set global options
\item
  Why style is important in coding
\item
  RPubs
\end{itemize}

\section{Week 6 (Quiz 6)}\label{week-6-quiz-6}

\begin{itemize}
\tightlist
\item
  Three characteristics of tidy data
\item
  Five common problems with tidy data and how to resolve them (make sure
  you understand the examples shown, which you can find out more about
  in the Hadley Wickham paper I reference)
\item
  \texttt{select}
\item
  \texttt{filter}
\item
  \texttt{mutate}
\item
  \texttt{summarize}
\item
  \texttt{group\_by}
\item
  \texttt{arrange}
\item
  \texttt{gather}
\item
  \texttt{spread}
\item
  The \texttt{*\_join} family of functions
\item
  \texttt{\%\textgreater{}\%}
\item
  Go through the examples from in-course exercises where we chained
  together several functions to clean up a dataset and make sure you can
  follow through these chained examples
\end{itemize}

\section{Week 7 (Quiz 7)}\label{week-7-quiz-7}

\begin{itemize}
\tightlist
\item
  \texttt{for} loops
\item
  basics of writing a function
\item
  figuring out the output of a loop based on its code
\item
  figuring the the output of a function based on its code
\item
  parentheses around a full assignment statement (e.g.,
  \texttt{(ex\ \textless{}-\ 1)})
\item
  \texttt{kable()} from the \texttt{knitr} package
\end{itemize}

\section{Week 8 (Quiz 8)}\label{week-8-quiz-8}

\begin{itemize}
\tightlist
\item
  \texttt{apply} family of functions
\item
  \texttt{matrix} objects, including how to subset
\item
  \texttt{list} objects, including how to subset
\end{itemize}

\chapter{Appendix B: Homework}\label{appendix-b-homework}

The following are six homework assignments for the course.

\section{Homework \#1}\label{homework-1}

\textbf{Due date: Sept. 14}

For your first homework assignment, you'll be working through a few
\href{http://swirlstats.com/}{swirl} lessons that are relevant to the
material we've covered so far. Swirl is a platform that helps you learn
R \textbf{in} R - you can complete the lessons right in your R console.

\subsection{Getting started}\label{getting-started}

First, you'll need to install the swirl package:

\begin{Shaded}
\begin{Highlighting}[]
\KeywordTok{install.packages}\NormalTok{(}\StringTok{"swirl"}\NormalTok{)}
\end{Highlighting}
\end{Shaded}

Next, load the swirl package. We're going to download a course from
swirl's \href{https://github.com/swirldev/swirl_courses}{course
repository} called R Programming E using the function
\texttt{install\_course\_github}. Then call the \texttt{swirl()}
function to enter the interactive platform:

\begin{Shaded}
\begin{Highlighting}[]
\KeywordTok{library}\NormalTok{(swirl)}
\KeywordTok{uninstall_course}\NormalTok{(}\StringTok{"R_Programming_E"}\NormalTok{) }\CommentTok{# Only run if you have an old version of}
                                    \CommentTok{# R_Programming_E installed}
\KeywordTok{install_course_github}\NormalTok{(}\StringTok{"swirldev"}\NormalTok{, }\StringTok{"R_Programming_E"}\NormalTok{)}
\KeywordTok{swirl}\NormalTok{()}
\end{Highlighting}
\end{Shaded}

\begin{rmdnote}
After calling \texttt{swirl()}, you may be prompted to clear your
workspace variables by running \texttt{rm=(list=ls())}. Running this
code will clear any variables you already have saved in your global
environment. While swirl recommends that you do this, it's not
necessary.
\end{rmdnote}

\subsection{Swirl lessons}\label{swirl-lessons}

Sign in with your name, and choose \emph{R Programming E} when swirl
asks you to choose a course. For this homework, you will need to work
through the following lessons in that course (the lesson number is in
parentheses):

\begin{itemize}
\tightlist
\item
  Basic Building Blocks (1)
\item
  Vectors (4)
\item
  Missing Values (5)
\item
  Subsetting Vectors (6)
\item
  Logic (8)
\item
  Looking at Data (12)
\item
  Dates and Times (14)
\end{itemize}

Each lesson should take about 10-15 minutes, but some are much shorter.
You can complete the lessons in any order you want, but you may find it
easiest to start with the lowest-numbered lessons and work your way up,
in the order we've listed the lessons here.

You'll be able to get started on some of these lessons after your first
day in class (Basic Building Blocks, for example), but others cover
topics that we'll get to in weeks 2 and 3. Whether or not we've covered
a swirl topic in class, you should be able to successfully work through
the lesson. At the end of each lesson, you'll be prompted to
``\texttt{inform\ someone\ about\ your\ successful\ completion\ of\ this\ lesson\ via\ email}.''
after answering \texttt{2} for `Yes,' enter your full name, and enter
\href{mailto:rachel.severson@colostate.edu}{\nolinkurl{rachel.severson@colostate.edu}}
as the email address of the person you'd like to notify. You should be
sending 7 emails in total.

\begin{rmdnote}
After telling swirl that you would like to send a notification email, an
already-populated email should pop up with the lesson you just completed
in the subject line - you just need to push send. This might not happen
if you access your email through a web browser instead of an app. In
this case, just send an email manually with a screenshot of the end of
the lesson, and the name of the lesson you just completed.
\end{rmdnote}

\subsection{Special swirl commands}\label{special-swirl-commands}

In the swirl environment, knowing about the following commands will be
helpful:

\begin{itemize}
\tightlist
\item
  Within each lesson, the prompt \texttt{...} indicates that you should
  hit Enter to move on to the next section.
\item
  \texttt{play()}: temporarily exit swirl. It can be useful during a
  swirl lesson to play around in the R console to try things out.
\item
  \texttt{nxt()}: regain swirl's attention after \texttt{play()}ing
  around in the console.
\item
  \texttt{main()}: return to swirl's main menu.
\item
  \texttt{bye()}: exit swirl. Swirl will save your progress if you exit
  in the middle of a lesson. You can also hit the Esc. key to exit. (To
  re-enter swirl, run \texttt{swirl()}. In a new R session you will have
  to first load the swirl library: \texttt{library(swirl)}.)
\end{itemize}

\subsubsection{For fun}\label{for-fun}

While they aren't required for class, you should consider trying out
some other swirl lessons later in the course. The \texttt{Functions}
lesson, as well as \texttt{lapply\ and\ sapply} and
\texttt{vapply\ and\ tapply} could be particularly useful. You can also
look through the \href{https://github.com/swirldev/swirl_courses}{course
directory} to see what other courses and lessons are available.

If you are doing extra swirl courses on your own, you probably want to
do them through the ``R Programming'', rather than the ``R Programming
E'', course, since you won't need to let us know by email. To get this,
you can run:

\begin{Shaded}
\begin{Highlighting}[]
\KeywordTok{library}\NormalTok{(swirl)}
\KeywordTok{install_course}\NormalTok{(}\StringTok{"R_Programming"}\NormalTok{)}
\KeywordTok{swirl}\NormalTok{()}
\end{Highlighting}
\end{Shaded}

\section{Homework \#2}\label{homework-2}

\textbf{Due date: Sept. 28}

For Homework 2, recreate the R Markdown document that you can download
from
\href{https://github.com/geanders/RProgrammingForResearch/raw/master/Homework/Homework_2.docx}{here}.

Here are some initial tips:

\begin{itemize}
\tightlist
\item
  Your goal is to create an R Markdown document that you can compile to
  create a Word document that looks just like the example document we've
  linked above.
\item
  You will turn in (by email) both the compiled Word document and the
  .Rmd original file.
\item
  Add your name as ``Author'' and the due date of the assignment as
  ``Date''. You should add these within the R Markdown document, rather
  than changing them in the final, compiled Word document.
\item
  If you want to get started before you know how to use R Markdown, you
  can go ahead and write all of the necessary code to replicate the
  output and figures in the document in an R script.
\item
  The code chunks here have been hidden with the option
  \texttt{echo\ =\ FALSE}, but you should include your code in your
  final document.
\item
  Set the chunk options \texttt{warning\ =\ FALSE} and
  \texttt{message\ =\ FALSE} to prevent warnings and messages from being
  printed out. You will get some messages and warnings in the code from
  things like missing values and from loading packages, but you want to
  hide all of those messages in your final document.
\item
  For things like templates, colors, level of transparency, and point
  size, you will receive full credit if you create figures that are
  visually similar to the ones shown in the example document. In other
  words, if the example document shows some transparency in points, you
  will get full credit if you also include some transparency in the
  points in your plot, but you do not have to include the exact same
  value of \texttt{alpha}.
\item
  In R, there are often many different ways to achieve something. As
  long as your code \emph{works}, it's fine if you haven't coded it
  exactly like we have in our version. However, your output should look
  identical to ours (or, in the case of color, transparency, point size,
  and themes, visually similar).
\item
  You will not lose points if you cannot recreate the table in the
  document (although you should try to!).
\item
  The last section, under the heading ``Extra challenge-- not graded'',
  is not graded. However, if you'd like an extra challenge, you're
  welcome to try it out and include it in your final submission!
\end{itemize}

If you need them, here are some further tips:

\begin{itemize}
\tightlist
\item
  Functions from the tidyverse (especially from \texttt{dplyr},
  \texttt{readr}, and \texttt{ggplot} packages) will make your life much
  easier for this exercise. You can now install and load the
  \texttt{tidyverse} package to load them all at once.
\item
  To rename column names with ``special'' characters in them, wrap the
  whole old column name in backticks. For example, to change a column
  name that has a dollar sign in it, you would use something like
  ``rename(new\_col\_name = `old\_col\_name\$`)''.
\item
  To change the size of a figure in a report, use the ``fig.width'' and
  ``fig.height'' chunk options.
\item
  You will want to use \texttt{scale\_fill\_brewer} in several of the
  figures.
\item
  Don't forget that, within functions like
  \texttt{scale\_x\_continuous}, you can use the argument
  \texttt{breaks} to set where the axis has breaks, and then
  \texttt{labels} to set what will actually be shown at each break.
\item
  The string ``\textbackslash{}n'' can be included in legends and labels
  to include a carriage return.
\item
  Coordinates can be flipped in a graph with the \texttt{coord\_flip}
  geom. So, if you can figure out a way to make a graph with the
  coordinates flipped, use that code and just add \texttt{coord\_flip}
  at the end.
\end{itemize}

\section{Homework \#3}\label{homework-3}

\textbf{Due date: Oct. 12}

For Homework 3, recreate the R Markdown document that you can download
from
\href{https://github.com/geanders/RProgrammingForResearch/raw/master/Homework/Homework_3.docx}{here}.

Here are some initial tips:

\begin{itemize}
\tightlist
\item
  Your goal is to create an R Markdown document that you can compile to
  create a Word document that looks just like the target document we've
  linked above. The only difference is that you will use
  \texttt{echo\ =\ TRUE} to show your code within the rendered Word
  document. All formating within the text should be similar or identical
  to the target document.\\
\item
  You will turn in (by email) both the compiled Word document and the
  .Rmd original file.
\item
  Add your name as ``Author'' and the due date of the assignment as
  ``Date''. You should add these within the R Markdown document, rather
  than changing them in the final, compiled Word document.
\item
  Set the chunk options \texttt{warning\ =\ FALSE} and
  \texttt{message\ =\ FALSE} to prevent warnings and messages from being
  printed out. You will get some messages and warnings in the code from
  things like missing values and from loading packages, but you want to
  hide all of those messages in your final document.
\item
  For things like templates, colors, level of transparency, and point
  size, you will receive full credit if you create figures that are
  visually similar to the ones shown in the example document. In other
  words, if the example document shows some transparency in points, you
  will get full credit if you also include some transparency in the
  points in your plot, but you do not have to include the exact same
  value of \texttt{alpha}.
\item
  In R, there are often many different ways to achieve something. As
  long as your code \emph{works}, it's fine if you haven't coded it
  exactly like we have in our version. However, your output should look
  identical to ours (or, in the case of color, transparency, point size,
  and themes, visually similar).
\item
  There is one formated table in the target document. Be sure that you
  render this as a formated table, not as raw R output.
\end{itemize}

If you need them, here are some further tips:

\begin{itemize}
\tightlist
\item
  Functions from the tidyverse (especially from \texttt{dplyr},
  \texttt{readr}, and \texttt{ggplot} packages) will make your life much
  easier for this exercise. You can now install and load the
  \texttt{tidyverse} package to load them all at once.
\item
  To reference column names with ``special'' characters in them, like
  dollar signs or spaces, wrap the whole old column name in backticks.
  For example, to change a column name that has a dollar sign in it, you
  would use something like ``rename(new\_col\_name =
  `old\_col\_name\$`)''.
\item
  To change the size of a figure in a report, use the ``fig.width'' and
  ``fig.height'' chunk options.
\item
  Don't forget that there are functions in the \texttt{scale} family
  that allow you to use log-scale axes.
\end{itemize}

\section{Homework \#4}\label{homework-4}

\textbf{Optional due date: Oct. 28}

All instructions for this homework can be downloaded
\href{https://github.com/geanders/RProgrammingForResearch/raw/master/Homework/Homework4and5.pdf}{here}.
The example ``fars\_analysis.pdf'' document you will try to recreate is
\href{https://github.com/geanders/RProgrammingForResearch/raw/master/Homework/fars_analysis.pdf}{here}.

You have the option to turn in parts of this homework (up through
creating a clean dataset) by Oct. 28. If you do so, I will email you the
code I used to clean the data, so you can check your own code and be
sure you have a reasonable version of the clean data as you do the final
parts of the assignment.

\section{Homework \#5}\label{homework-5}

\textbf{Due date: Nov. 9}

All instructions for this homework can be downloaded
\href{https://github.com/geanders/RProgrammingForResearch/raw/master/Homework/Homework4and5.pdf}{here}.
The example ``fars\_analysis.pdf'' document you will try to recreate is
\href{https://github.com/geanders/RProgrammingForResearch/raw/master/Homework/fars_analysis.pdf}{here}.

You will submit this homework by posting a repo with your project
directory on GitHub. We will work on setting that up during an in-course
exercise.

\section{Homework \#6}\label{homework-6}

\textbf{Due date: Nov. 30}

\begin{enumerate}
\def\labelenumi{\arabic{enumi}.}
\item
  Read the article \emph{Good Enough Practices in Scientific Computing}
  by Wilson et al. (available
  \href{https://arxiv.org/abs/1609.00037}{here}). In a half page,
  describe which of these ``pretty good practices'' your last homework
  incorporated. Also list one or two practices that you did not follow
  in your last homework but that would have made sense and how you could
  have followed them.
\item
  Read the article
  \href{http://fivethirtyeight.com/features/science-isnt-broken/}{\emph{Science
  Isn't Broken}} on FiveThirtyEight. This article includes an
  interactive graphic. In a half page, give your opinion on whether this
  interactive graphic helps convey the main message of the article.
  Also, describe in general details how you might be able to create a
  graphic like this in R.
\item
  Find an article in \emph{The R Journal} that describes an R package
  that you could use in your own research or otherwise find interesting.
  Describe why the package was created and what you think it's most
  interesting features are. In an R Markdown document, run one or two of
  the R examples included in the article.
\end{enumerate}

\bibliography{packages.bib,book.bib}

\backmatter
\printindex

\end{document}
